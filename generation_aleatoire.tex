Afin de générer aléatoirement des chaînes de taille $n$, nous créons la fonction \textit{random\_generate}, qui comme son nom l'indique, se repose sur la fonction rand() du langage C++ :

\begin{center}
\begin{minipage}{0.9\textwidth}
\begin{center}
\begin{lstlisting}
  //Generate a random ADM sequence (size = n)
// O(n) - Need to free the result
char* random_generate(const int &n,int randomvalue)
{
  srand(time(NULL)+randomvalue);
  char ADN[4]={'A','T','C','G'};
  char* result=(char*) malloc((n+2)*sizeof(char));
  int i;
  result[0]='x';
  for(i=0;i<n;i++)
  {
    result[i+1]=ADN[rand()%4];
  }
  result[i+1]='\0';
  return result;
}
\end{lstlisting}
\end{center}
\end{minipage} 
\end{center}

La présence de l'entier \textbf{randomvalue} réside dans le fait que la fonction random n'est pas totalement aléaoire, et dépend en partie de l'heure système (en seconde).
Ainsi, en appelant plusieurs fois la fonction dans la même seconde, les chaînes retournées seront toujours identiques. Il faut donc penser à incrémenter le compteur à chaque génération, et conserver la base de temps initial\footnote{À voir dans l'algorithme principal}.
