\part{Les algorithmes}
\section{Choix d'impl�mentation}
  \subsection{Algorithme 1}
  Pour implementer l'algorithme 1, nous avons choisi de repr�senter Mlsuff comme un int**.\\
  
  Le remplissage de la matrice se fait en deux temps : \begin{itemize}
  \item On remplit la premiere ligne et la premiere colonne de la matrice, qui revient � comparer le premier caract�re de chaque tableau avec chacun de ceux de l'autre tableau.
  \item On remplit le reste du tableau gr�ce � la formule liant Mlsuff\[i\]\[j\] � Mlsuff\[i+1\]\[j+1\].
  \end{itemize}
  \vspace{0.5cm}
  
  on reparcourt ensuite chaque case du tableau pour conserver le maximum que l'on renvoie en fin de parcours.\\
  
  \subsection{Algorithme 2}
  
  \subsection{Algorithme 3}
  
\section{Complexit� th�orique}
  \subsection{Algorithme 1}
  l'algorithme n�cessite de parcourir 2 fois une matrice de taille n par n, ce qui �quivaut � une complexit� en $O(n^2)$ pour les parcours.\\

  Les m�mes op�rations sont effectu�es sur chaque case, elles sont donc ind�pendantes de n, ce sont des op�rations d'affectation, de test d'�galit� et d'addition, c'est � dire toutes en $O(1)$.\\

  La complexit� reste donc en $O(n^2)$.
  
  \subsection{Algorithme 2}
  
  \subsection{Algorithme 3}