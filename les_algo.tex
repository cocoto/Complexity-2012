\chapter{Les algorithmes}
\section{Choix d'implémentation}
  
  Nous avons choisi d'implémenter ces algorithmes en C++, langage à la fois simple pour ce que l'on veut faire et efficace.
  
  \subsection{Algorithme 1}
  Pour implementer l'algorithme 1, nous avons choisi de représenter Mlsuff comme un int**.\\
  
  Le remplissage de la matrice se fait en deux temps : \begin{itemize}
  \item On remplit la premiere ligne et la premiere colonne de la matrice, qui revient à comparer le premier caractère de chaque tableau avec chacun de ceux de l'autre tableau.
  \item On remplit le reste du tableau grâce à la formule liant Mlsuff$[i][j]$ à Mlsuff$[i+1][j+1]$.
  \end{itemize}
  \vspace{0.5cm}
  
  on reparcourt ensuite chaque case du tableau pour conserver le maximum que l'on renvoie en fin de parcours.\\
  
  \subsection{Algorithme 2}
  
	Cet algorithme est implémenté par 3 boucles \textbf{for} imbriquées : \begin{itemize}
\item Une pour traiter toutes les tailles de 1 à n.
\item Une pour choisir chaque sous-chaîne de la taille en cours dans S.
\item Une pour choisir chaque sous-chaîne de la taille en cours dans T.
 \end{itemize}
 \vspace{0.5cm}
 
 Au sein de la derniere boucle, on effectue la comparaison de la paire de chaîne en cours et si la condition est respectée on affecte la nouvelle valeur à {\textbf max}.
 
  \subsection{Algorithme 3}
	
	Cet algorithme est principalement codé comme une boucle {\textbf tant que} pour le remplissage d'un tableau PMK (int*), et d'une boucle {\textbf for} pour la comparaison avec T.\\
	
	Ces étapes sont éxécutées systématiquement au sein de 2 boucles {\textbf for}.\\
	
	Le tableau $S_{ij}$ n'est jamais créé, on se contente d'acceder à la case de S en opérant un décalage en fonction du i et du j actuel. On évite ainsi de créer et recopier le tableau pour chaque paire de i j.
	
\section{Complexité théorique}
  \label{c_theorique}
  \subsection{Algorithme 1}
  l'algorithme nécessite de parcourir 2 fois une matrice de taille n par n, ce qui équivaut à une complexité en $O(n^2)$ pour les parcours.\\

  Les mêmes opérations sont effectuées sur chaque case, elles sont donc indépendantes de n, ce sont des opérations d'affectation, de test d'égalité et d'addition, c'est à dire toutes en $O(1)$.\\

  La complexité reste donc en $O(n^2)$.
  
  \subsection{Algorithme 2}
	Cet algorithme contient 3 boucles imbriquées dont la longueur dépend de la taille n des données (sans être égale à n).\\
	
	A l'interieur de ces boucles, on effectue la comparaison de deux chaînes de taille l ( $1 \leq l \leq n$ ), ce qui donne une opération de complexité $O(l)$, soit au pire $O(n)$.\\
	
	Nous obtenons donc une complexité en $O(n^4)$
	
	
  \subsection{Algorithme 3}
  
	L'algorithme s'exécute pour toute paire i,j avec ($ 1 \leq i \leq j \leq n$). Cela revient à une boucle de complexité $O(n^2)$.\\

	La première partie de l'algorithme est une boucle {\textbf tant que}, {\textbf a} augmentant régulièrement et {\textbf sij } allant de 1 à n, on obtient une complexité $O(n)$ pour cette boucle. De plus tout ce que s'effectue au sein de la boucle est en $O(1)$ y compris la boucle des lignes 6-8.\\

	La seconde partie est une boucle {\textbf Pour } de taille n. De même tout ce qui s'effectue à l'interieur est en $O(1)$.\\
	
	Le nombre d'opérations nécessaires est donc en $O(n)$ au sein d'une boucle en $O(n^2)$.
	
	On a donc une complexité globale en $O(n^3)$.
	