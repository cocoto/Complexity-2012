Afin de générer l'ensemble des chaînes possibles de taille n pour un alphabet $\{A,B,C,D\}$, nous avons eu recours à un script en langage \textbf{prolog} (voir Annexe \ref{code_prolog}).
Ce langage permet de facilement écrire des scripts usant (et abusant) du backtracking. Le script choisi une lettre à la position $i$ ($i\in \{1..n\})$, créé un point de choix, et passe à la case suivante.
Une fois le mot formé, l'algorithme effectue un backtrack sur ses points de choix et génère ainsi l'ensemble des chaînes possibles. On utilise la redirection de sortie standard (commande \textit{>} en shell linux) de ce programme vers un fichier.
On obtient par exemple sur une taille $n=2$ le fichier suivant :\\
\begin{center}
 \fbox{
\begin{minipage}{0.9\textwidth}\centering
 2
AA
AT
AC
AG
TA
TT
TC
TG
CA
CT
CC
CG
GA
GT
GC
GG
\end{minipage}
}
\end{center}

Comparer toutes les chaînes générées, nous simulons la présence de deux curseurs dans le fichier, le premier parcourant les chaînes S, et le second toutes les suivantes (T).