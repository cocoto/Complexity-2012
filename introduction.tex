La complexité temporelle d'un algorithme est primordiale lorsque l'on souhaite traiter un grand volume de données.
Le but de ce projet est de comparer trois algorithmes effectuant la même opération : comparer deux chaînes de caractères de tailles identiques afin d'en retourner la taille de la plus longue sous-chaîne commune.
Ces algorithmes trouvent application dans de nombreux domaines, comme par exemple la génétique (comparaison de chaînes A.D.N.), ou dans la cryptographie (étude de fréquence des mots ou des phrases dans deux textes encodés de la même manière).
Nous implémenterons donc les trois algorithmes $A_1$,$A_2$ et $A_3$ donnés dans le cadre du cours, et nous essayerons d'analyser la complexité de ces fonctions
\begin{itemize}
 \item D'une part théoriquement (en analysant le pseudo-code donné)
 \item D'autre part expérimentalement (en chronométrant les temps d'exécution par des jeux de tests bien choisis, puis en extrapolant les résultats obtenus pour retrouver la fonction associée)
\end{itemize}

Dans ce projet, nous ne nous intéresserons pas à la complexité en espace des trois algorithmes, mais les contraintes de matériel révélerons éventuellement quelques informations sur la nature spaciale des fonctions.